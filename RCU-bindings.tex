\documentclass[letterpaper,10pt]{article}
\usepackage{epsfig,endnotes}
%\usepackage{usenix,epsfig}

%%%%%%%%%%%%%%%%%%%%%%%%%%%%%%%%%%%%%%%%%%%%%%%%%%%%%%%%%%%%%%%%%%%%%%
\usepackage{subfig}
\usepackage{fixltx2e}
\usepackage{url}        % \url{} command with good linebreaks
\usepackage{amssymb,amsmath}
\usepackage{graphicx}
\usepackage{enumerate}
\usepackage{listings}
\usepackage{xspace}
\usepackage[table]{xcolor}
\usepackage{rotating}
\lstset{basicstyle=\ttfamily}
\usepackage[bookmarks=true,bookmarksnumbered=true,pdfborder={0 0 0}]{hyperref}

% Avoid widows and orphans
\widowpenalty=500
\clubpenalty=500

%%%%%%%%%%%%%%%%%%%%%%%%%%%%%%%%%%%%%%%%%%%%%%%%%%%%%%%%%%%%%%%%%%%%%%

\begin{document}

\lstset{
	literate={\_}{}{0\discretionary{\_}{}{\_}}%
}
\newcommand{\co}[1]{\lstinline[breaklines=yes,breakatwhitespace=yes]{#1}}

\title{P0461R1: Proposed RCU C++ API}

\author{
{\bf Doc. No.: } WG21/P0461R1 \\
{\bf Date: } 2017-02-07 \\
{\bf Reply to: } Paul E. McKenney, Maged Michael, Michael Wong,\\
Isabella Muerte, Arthur O'Dwyer, and David Hollman \\
{\bf Email: } paulmck@linux.vnet.ibm.com, maged.michael@gmail.com, \\
fraggamuffin@gmail.com, isabella.muerte@mnmlstc.com, \\
arthur.j.odwyer@gmail.com, and dshollm@sandia.gov
} % end author

% Use the following at camera-ready time to suppress page numbers.
% Comment it out when you first submit the paper for review.
%\thispagestyle{empty}

\pagestyle{myheadings}
\markright{WG21/P0461R1}

\maketitle

%%%%%%%%%%%%%%%%%%%%%%%%%%%%%%%%%%%%%%%%%%%%%%%%%%%%%%%%%%%%%%%%%%%%%%

This document is based on WG21/P0279R1 combined with feedback at
the 2015 Kona and 2016 Jacksonville meetings, which most notably called
for a C++-style method of handling different RCU implementations or
domains within a single translation unit, and which also contains
useful background material and references.
Unlike WG21/P0279R1, which simply introduced RCU's C-language practice,
this document presents proposals for C++-style RCU APIs.
At present, it appears that these are not conflicting proposals, but
rather ways of handling different C++ use cases resulting from
inheritance, templates, and different levels of memory pressure.
This document also incorporates content from
WG21/P0232R0\cite{PaulEMcKennneyToolKitP0232R0}.

Note that this proposal is related to the hazard-pointer proposal in
that both proposals defer destructive actions such as reclamation until
all readers have completed.
See P0233R3, which updates ``P0233R2: Hazard Pointers:
Safe Resource Reclamation for Optimistic Concurrency''
at \url{http://www.open-std.org/jtc1/sc22/wg21/docs/papers/2016/p0233r2.pdf}.

Note also that a redefinition of the infamous \co{memory_order_consume}
is the subject of two separate papers:

\begin{enumerate}
\item	P0190R3, which updates
	``P0190R2: Proposal for New \co{memory_order_consume} Definition'',
	\url{http://www.open-std.org/jtc1/sc22/wg21/docs/papers/2016/p0190r2.pdf}.
\item	P0462R1, which updates
	``P0462R0: Marking \co{memory_order_consume} Dependency Chains'',
	\url{http://www.open-std.org/jtc1/sc22/wg21/docs/papers/2016/p0462r0.pdf}.
\end{enumerate}

Draft wording for this proposal may be found in the new working paper
``P0566R0: Proposed Wording for Concurrent Data Structures: Hazard
Pointer and Read-Copy-Update (RCU)''.

A detailed change log appears starting on
page~\pageref{sec:Change Log}.

\section{Introduction}
\label{sec:Introduction}

This document proposes C++ APIs for read-copy update (RCU).
For more information on RCU, including RCU semantics, see
WG21/P0462R0 (``Marking \co{memory_order_consume} Dependency Chains''),
WG21/P0279R1 (``Read-Copy Update (RCU) for C++''),
WG21/P0190R2 (``Proposal for New \co{memory_order_consume} Definition''),
and
WG21/P0098R1 (``Towards Implementation and Use of \co{memory_order_consume}'').

Specifically, this document proposes
\co{rcu_domain} (Figure~\ref{fig:RCU Domain Base Class}),
\co{rcu_guard} (Figure~\ref{fig:RCU Guarded Readers}), and
\co{rcu_obj_base} (Figure~\ref{fig:RCU Callbacks: Derived-Type Approach}).

Section~\ref{sec:Existing C-Language RCU API} presents the base (C-style) RCU API,
Section~\ref{sec:RCU Guarded Readers} presents a proposal for scoped RCU readers,
Section~\ref{sec:RCU Callback Handling} presents proposals for handling of
RCU callbacks (with that of
Section~\ref{sec:Derived-Type Approach} being the preferred implementation),
Section~\ref{sec:Hazard Pointers and RCU: Which to Use?} presents a
table comparing reference counting, hazard pointers, and RCU, and finally
Section~\ref{sec:Summary} presents a summary.

\section{Existing C-Language RCU API}
\label{sec:Existing C-Language RCU API}

\begin{figure}[tbp]
{ \scriptsize
\begin{verbatim}
 1 void std::rcu_read_lock();
 2 void std::rcu_read_unlock();
 3 void std::synchronize_rcu();
 4 void std::call_rcu(struct std::rcu_head *rhp,
 5                    void cbf(class rcu_head *rhp));
 6 void std::rcu_barrier();
 7 void std::rcu_register_thread();
 8 void std::rcu_unregister_thread();
 9 void std::rcu_quiescent_state();
10 void std::rcu_thread_offline();
11 void std::rcu_thread_online();
\end{verbatim}
}
\caption{Existing C-Language RCU API}
\label{fig:Existing C-Language RCU API}
\end{figure}

Figure~\ref{fig:Existing C-Language RCU API}
shows the existing C-language RCU API as provided by implementations such as
userspace RCU~\cite{MathieuDesnoyers2009URCU,PaulMcKenney2013LWNURCU}.
This API is provided for compatibility with existing C-language practice as
well as to provide the highest performance for fast-path code.
(See Figure~\ref{fig:RCU Domain Base Class} for a proposed API that
permits multiple RCU domains, as requested by several committee members.)

Lines~1 and~2 show \co{rcu_read_lock()} and \co{rcu_read_unlock()},
which mark the beginning and the end, respectively, of an \emph{RCU read-side
critical section}.
These primitives may be nested, and matching \co{rcu_read_lock()}
and \co{rcu_read_unlock()} calls need not be in the same scope.
(That said, it is good practice to place them in the same scope
in cases where the entire critical section fits comfortably into
one scope.)

Line~3 shows \co{synchronize_rcu()}, which waits for any pre-existing
RCU read-side critical sections to complete.
The period of time that \co{synchronize_rcu()} is required to wait is
called a \emph{grace period}.
Note that a given call to \co{synchronize_rcu()} is \emph{not} required to
wait for critical sections that start later.

Lines~4 and~5 show \co{call_rcu()}, which, after a subsequent grace period
elapses, causes the \co{cbf(rhp)} \emph{RCU callback function} to be invoked.
Thus, \co{call_rcu()} is the asynchronous counterpart to
\co{synchronize_rcu()}.
In most cases, \co{synchronize_rcu()} is easier to use, however, \co{call_rcu()}
has the benefit of moving the grace-period delay off of the updater's
critical path.
Use of \co{call_rcu()} is thus critically important for good performance of
update-heavy workloads, as has been repeatedly discovered by any number of
people new to RCU~\cite{PaulEMcKenney2015ReadMostly}.

Note that although \co{call_rcu()}'s callbacks are guaranteed not to be
invoked too early, there is no guarantee that their execution won't be
deferred for a considerable time.
This can be a problem if a given program requires that all outstanding
RCU callbacks be invoked before that program terminates.
The \co{rcu_barrier()} function shown on line~6 is intended for this
situation.
This function blocks until all callbacks corresponding to previous
\co{call_rcu()} invocations have been invoked and also until after
those invocations have returned.
Therefore, taking the following steps just before terminating a program
will guarantee that all callbacks have completed:

\begin{enumerate}
\item	Take whatever steps are required to ensure that there are no
	further invocations of \co{call_rcu()}.
\item	Invoke \co{rcu_barrier()}.
\end{enumerate}

Carrying out this procedure just prior to program termination can be very
helpful for avoiding false positives when using tools such as valgrind.

Many RCU implementations require that every thread announce itself to
RCU prior to entering the first RCU read-side critical section, and
to announce its departure after exiting the last RCU read-side
critical section.
These tasks are carried out via the \co{rcu_register_thread()} and
\co{rcu_unregister_thread()}, respectively.

The implementations of RCU that feature the most aggressive implementations of
\co{rcu_read_lock()} and \co{rcu_read_unlock()} require that each thread
periodically pass through a \emph{quiescent state}, which is announced to RCU
using \co{rcu_quiescent_state()}.
A thread in a quiescent state is guaranteed not to be in an RCU
read-side critical section.
Threads can also announce entry into and exit from \emph{extended
quiescent states}, for example, before and after blocking system
calls, using \co{rcu_thread_offline()} and \co{rcu_thread_online()}.

\subsection{RCU Domains}
\label{sec:RCU Domains}

The userspace RCU library features several RCU implementations, each
optimized for different use cases.

The quiescent-state based reclamation (QSBR) implementation is intended
for standalone applications where the developers have full control
over the entire application, and where extreme read-side performance
and scalability is required.
Applications use \co{#include "urcu-qsbr.hpp"} to select QSBR and
\co{-lurcu -lurcu-qsbr} to link to it.
These applications must use \co{rcu_register_thread()} and
\co{rcu_unregister_thread()} to announce the coming and going
of each thread that is to execute \co{rcu_read_lock()} and
\co{rcu_read_unlock()}.
They must also use \co{rcu_quiescent_state()}, \co{rcu_thread_offline()},
and \co{rcu_thread_online()} to announce quiescent states to RCU.

The memory-barrier implementation is intended for applications that
can announce threads (again using \co{rcu_register_thread()} and
\co{rcu_unregister_thread()}), but for which announcing quiescent states is
impractical.
Such applications use \co{#include "urcu-mb.hpp"} and
\co{-lurcu-mb} to select the memory-barrier implementation.
Such applications will incur the overhead of a full memory barrier in
each call to \co{rcu_read_lock()} and \co{rcu_read_unlock()}.

The signal-based implementation represents a midpoint between the QSBR
and memory-barrier implementations.
Like the memory-barrier implementation, applications must announce
threads, but need not announce quiescent states.
On the one hand, readers are almost as fast as in the QSBR implementation,
but on the other applications must give up a signal to RCU, by default
\co{SIGUSR1}.
Such applications use \co{#include "urcu-signal.hpp"} and
\co{-lurcu-signal} to select signal-based RCU.

So-called ``bullet-proof RCU'' avoids the need to announce either threads
or quiescent states, and is therefore the best choice for use by
libraries that might well be linked with RCU-oblivious applications.
The penalty is that \co{rcu_read_lock()} incurs both a memory barrier
and a test and \co{rcu_read_unlock()} incurs a memory barrier.
Such applications or libraries use \co{#include urcu-bp.hpp} and
\co{-lurcu-bp}.

\subsection{Run-Time Domain Selection}
\label{sec:Run-Time Domain Selection}

\begin{figure}[tbp]
{ \scriptsize
\begin{verbatim}
 1 class rcu_domain {
 2 public:
 3   constexpr explicit rcu_domain() noexcept { };
 4   rcu_domain(const rcu_domain&) = delete;
 5   rcu_domain(rcu_domain&&) = delete;
 6   rcu_domain& operator=(const rcu_domain&) = delete;
 7   rcu_domain& operator=(rcu_domain&&) = delete;
 8   virtual void register_thread() = 0;
 9   virtual void unregister_thread() = 0;
10   static constexpr bool register_thread_needed() { return true; }
11   virtual void quiescent_state() noexcept = 0;
12   virtual void thread_offline() noexcept = 0;
13   virtual void thread_online() noexcept = 0;
14   static constexpr bool quiescent_state_needed() { return false; }
15   virtual void read_lock() noexcept = 0;
16   virtual void read_unlock() noexcept = 0;
17   virtual void synchronize() noexcept = 0;
18   virtual void retire(rcu_head *rhp, void (*cbf)(rcu_head *rhp)) = 0;
19   virtual void barrier() noexcept = 0;
20 };
\end{verbatim}
}
\caption{RCU Domain Base Class}
\label{fig:RCU Domain Base Class}
\end{figure}

Figure~\ref{fig:RCU Domain Base Class}
shows the abstract base class for runtime selection of RCU domains.
Each domain creates a concrete subclass that implements its RCU APIs:

\begin{itemize}
\item	Bullet-proof RCU: \co{class rcu_bp}
\item	Memory-barrier RCU: \co{class rcu_mb}
\item	QSBR RCU: \co{class rcu_qsbr}
\item	Signal-based RCU: \co{class rcu_signal}
\end{itemize}

Of course, additional implementations of RCU may be constructed by
deriving from \co{rcu_domain} and/or by implementing the API
shown in
Figure~\ref{fig:Existing C-Language RCU API}.

\section{RCU Guarded Readers}
\label{sec:RCU Guarded Readers}

\begin{figure}[tbp]
{ \scriptsize
\begin{verbatim}
 1   class rcu_guard {
 2   public:
 3     rcu_guard() noexcept
 4     {
 5       this->rd = nullptr;
 6       rcu_read_lock();
 7     }
 8
 9     explicit rcu_guard(rcu_domain *rd)
10     {
11       this->rd = rd;
12       rd->read_lock();
13     }
14
15     rcu_guard(const rcu_guard &) = delete;
16
17     rcu_guard&operator=(const rcu_guard &) = delete;
18
19     ~rcu_guard() noexcept
20     {
21       if (this->rd)
22         this->rd->read_unlock();
23       else
24         rcu_read_unlock();
25     }
26
27   private:
28     rcu_domain *rd;
29   };
\end{verbatim}
}
\caption{RCU Guarded Readers}
\label{fig:RCU Guarded Readers}
\end{figure}

In some cases, it might be convenient to use a guard style for RCU readers,
especially if the read-side critical section might be exited via exception.
The \co{rcu_guard} class shown in
Figure~\ref{fig:RCU Guarded Readers}
may be used for this purpose.
An argumentless constructor uses the API, or an \co{rcu_domain} class
may be passed to the constructor to use the specified RCU
implementation.

This is intended to be used in a manner similar to
\co{std::lock_guard}.

\section{RCU Callback Handling}
\label{sec:RCU Callback Handling}

The traditional C-language RCU callback uses address arithmetic
to map from the \co{rcu_head} structure to the enclosing struct,
for example, via the \co{container_of()} macro.
Of course, this approach also works for C++, but this section first
looks at some approaches that leverage C++ overloading and inheritance,
which has the benefit of avoiding macros and providing better type safety.
This will not be an either-or situation: Several of these approaches
are likely to be generally useful.

However, the approach discussed in
Section~\ref{sec:Derived-Type Approach}
is the preferred approach, and is compatible with the proposal for
hazard pointers.
The approaches in the other sections are presented for informational
purposes:
Section~\ref{sec:Pointer To Enclosing Class}
uses a pointer to an enclosing type and
Section ~\ref{sec:Address Arithmetic}
uses address arithmetic (illustrating the C-language approach used in
the Linux kernel).

\subsection{Derived-Type Approach (Preferred)}
\label{sec:Derived-Type Approach}

\begin{figure}[tbp]
{ \scriptsize
\begin{verbatim}
 1   template<typename T,
 2            typename D = default_delete<T>,
 3            bool E = is_empty<D>::value>
 4   class rcu_obj_base: private rcu_head {
 5     D deleter;
 6   public:
 7     static void trampoline(rcu_head *rhp)
 8     {
 9       auto rhdp = static_cast<rcu_obj_base *>(rhp);
10       auto obj = static_cast<T *>(rhdp);
11       rhdp->deleter(obj);
12     }
13
14     void retire(D d = {})
15     {
16       deleter = d;
17       call_rcu(static_cast<rcu_head *>(this), trampoline);
18     }
19
20     void retire(rcu_domain &rd, D d = {})
21     {
22       deleter = d;
23       rd.retire(static_cast<rcu_head *>(this), trampoline);
24     }
25   };
26
27   template<typename T, typename D>
28   class rcu_obj_base<T,D,true>: private rcu_head {
29   public:
30     static void trampoline(rcu_head *rhp)
31     {
32       auto rhdp = static_cast<rcu_obj_base *>(rhp);
33       auto obj = static_cast<T *>(rhdp);
34       D()(obj);
35     }
36
37     void retire(D d = {})
38     {
39       call_rcu(static_cast<rcu_head *>(this), trampoline);
40     }
41
42     void retire(rcu_domain &rd, D d = {})
43     {
44       rd.retire(static_cast<rcu_head *>(this), trampoline);
45     }
46   };
\end{verbatim}
}
\caption{RCU Callbacks: Derived-Type Approach (Preferred)}
\label{fig:RCU Callbacks: Derived-Type Approach}
\end{figure}

The \co{rcu_obj_base} class provides overloaded \co{retire()} methods,
as shown in
Figure~\ref{fig:RCU Callbacks: Derived-Type Approach}.
These methods take a deleter and an optional
\co{rcu_domain} class instance.
The deleter's operator() is invoked after a grace period.
The deleter type defaults to \co{std::default_delete<T>},
but one could also use a
custom functor class with an \co{operator()} that carries out teardown actions
before freeing the object, or a raw function pointer type such as
\co{void(*)(T*)}, or a lambda type.
We recommend avoiding deleter types such as \co{std::function<void(T*)>}
(and also any other type requiring memory allocation) because
allocating memory on the free path can result in out-of-memory deadlocks.

If an \co{rcu_domain} is supplied, its \co{retire()}
member function is used, otherwise the \co{call_rcu()} free
function is used.

Please note that this is the preferred approach, and is being proposed
for standardization.

The next section provides a specialization that only permits \co{delete},
which allows omitting the deleter, thus saving a bit of memory.

\subsection{Pointer To Enclosing Class (Informational Only)}
\label{sec:Pointer To Enclosing Class}

\begin{figure}[tbp]
{ \scriptsize
\begin{verbatim}
 1   template<typename T>
 2   class rcu_head_ptr: public rcu_head {
 3   public:
 4     rcu_head_ptr()
 5     {
 6       this->container_ptr = nullptr;
 7     }
 8
 9     rcu_head_ptr(T *containing_class)
10     {
11       this->container_ptr = containing_class;
12     }
13
14     static void trampoline(rcu_head *rhp)
15     {
16       T *obj;
17       rcu_head_ptr<T> *rhdp;
18
19       rhdp = static_cast<rcu_head_ptr<T> *>(rhp);
20       obj = rhdp->container_ptr;
21       if (rhdp->callback_func)
22         rhdp->callback_func(obj);
23       else
24         delete obj;
25     }
26
27     void retire(void callback_func(T *obj) = nullptr)
28     {
29       this->callback_func = callback_func;
30       call_rcu(static_cast<rcu_head *>(this), trampoline);
31     }
32
33     void retire(class rcu_domain &rd,
34         void callback_func(T *obj) = nullptr)
35     {
36       this->callback_func = callback_func;
37       rd.retire(static_cast<rcu_head *>(this), trampoline);
38     }
39
40   private:
41     void (*callback_func)(T *obj);
42     T *container_ptr;
43   };
\end{verbatim}
}
\caption{RCU Callbacks: Pointer (Informational Only)}
\label{fig:RCU Callbacks: Pointer}
\end{figure}

If complex inheritance networks make inheriting from an
\co{rcu_head} derived type impractical, one alternative is
to maintain a pointer to the enclosing class as shown in
Figure~\ref{fig:RCU Callbacks: Pointer}.
This \co{rcu_head_ptr} class is included as a member of the RCU-protected
class.
The \co{rcu_head_ptr} class's pointer must be initialized, for example,
in the RCU-protected class's constructor.

If the RCU-protected class is \co{foo} and the name of the
\co{rcu_head_ptr} member function is \co{rh}, then
\co{foo1.rh.retire(my_cb)} would cause the function \co{my_cb()} to be
invoked after the end of a subsequent grace period.
As with the previous classes, omitting the deleter results
in the object being passed to \co{delete} and an \co{rcu_domain}
object may be specified.

Please note that this section is informational only: This approach
is \emph{not} being proposed for standardization.

\subsection{Address Arithmetic (Informational Only)}
\label{sec:Address Arithmetic}

\begin{figure}[tbp]
{ \scriptsize
\begin{verbatim}
 1   template<typename T>
 2   class rcu_head_container_of {
 3   public:
 4     static void set_field(const struct rcu_head T::*rh_field)
 5     {
 6       T t;
 7       T *p = &t;
 8
 9       rh_offset = ((char *)&(p->*rh_field)) - (char *)p;
10     }
11
12     static T *enclosing_class(struct rcu_head *rhp)
13     {
14       return (T *)((char *)rhp - rh_offset);
15     }
16
17   private:
18     static inline size_t rh_offset;
19   };
20
21   template<typename T>
22   size_t rcu_head_container_of<T>::rh_offset;
\end{verbatim}
}
\caption{RCU Callbacks: Address Arithmetic (Informational Only)}
\label{fig:RCU Callbacks: Address Arithmetic}
\end{figure}

Figure~\ref{fig:RCU Callbacks: Address Arithmetic}
shows an approach that can be used if memory is at a premium and
the inheritance techniques cannot be used.
The \co{set_field()} method sets the offset of the
\co{rcu_head_container_of} member within the enclosing RCU-protected
structure, and the \co{enclosing_class()} member function
applies that offset to translate a pointer to the
\co{rcu_head_container_of} member to the enclosing RCU-protected structure.

\begin{figure*}[tbp]
{ \scriptsize
\begin{verbatim}
 1 void my_cb(struct std::rcu_head *rhp)
 2 {
 3   struct foo *fp;
 4
 5   fp = std::rcu_head_container_of<struct foo>::enclosing_class(rhp);
 6   std::cout << "Callback fp->a: " << fp->a << "\n";
 7 }
\end{verbatim}
}
\caption{RCU Callbacks: Address Arithmetic in Callback (Informational Only)}
\label{fig:RCU Callbacks: Address Arithmetic in Callback}
\end{figure*}

This address arithmetic must be carried out in the callback function,
as shown in
Figure~\ref{fig:RCU Callbacks: Address Arithmetic in Callback}.

Please note that this section is informational only: This approach
is \emph{not} being proposed for standardization.

\section{Hazard Pointers and RCU: Which to Use?}
\label{sec:Hazard Pointers and RCU: Which to Use?}

\begin{sidewaystable*}
\small
\centering
\begin{tabular}{p{1.2in}|p{1.2in}|p{1.2in}|p{1.2in}|p{1.2in}}
	& Reference Counting
		& \raggedright Reference Counting with DCAS
			& RCU
				& Hazard Pointers \\
	\hline
	\hline
	Unreclaimed objects
		& \cellcolor{green!30}{\bf Bounded}
			& \cellcolor{green!30}{\bf Bounded}
				& Unbounded
					& \cellcolor{green!30}{\bf Bounded} \\
	\hline
	\raggedright
	Contention among readers
		& Can be very high
			& Can be very high
				& \cellcolor{green!30}{\bf No contention}
					& \cellcolor{green!30}
					  {\bf No contention} \\
	\hline
	\raggedright
	Traversal forward progress
		& Either blocking or lock-free with limited reclamation
			& \cellcolor{green!30} {\bf Lock free}
				& \cellcolor{blue!20}
				  {\bf Bounded population oblivious wait-free}
					& \cellcolor{green!30}{\bf Lock-free} \\
	\hline
	\raggedright
	Reclamation forward progress $^*$
		& Either blocking or lock-free with limited reclamation
			& \cellcolor{green!30} {\bf Lock free}
				& Blocking
					& \cellcolor{blue!20}
					  {\bf Bounded wait-free} \\
	\hline
	Traversal speed
		& Atomic read-modify-write updates
			& Atomic read-modify-write updates
				& \cellcolor{green!30}{\bf No or low overhead}
					& Store-load fence \\
	\hline
	Reference acquisition
		& \cellcolor{green!30}{\bf Unconditional}
			& \cellcolor{green!30}{\bf Unconditional}
				& \cellcolor{green!30}{\bf Unconditional}
					& Conditional \\
	\hline
	\raggedright
	Automatic reclamation
		& \cellcolor{green!30}{\bf Yes}
			& \cellcolor{green!30}{\bf Yes}
				& No
					& No \\
	\hline
	Purpose of domains
		& N/A
			& N/A
				& Isolate long-latency readers
					& Limit contention, reduce space
					  bounds, etc.  \\
\end{tabular}
\caption{Comparison of Deferred-Reclamation Mechanisms}
\label{tab:Comparison of Deferred-Reclamation Mechanisms}

\flushleft
\noindent
*~~Does not include memory allocator, just the reclamation itself.
\end{sidewaystable*}

Table~\ref{tab:Comparison of Deferred-Reclamation Mechanisms}
provides a rough summary of the relative advantages of reference
counting, RCU, and hazard pointers.
Advantages are marked in bold with green background, or with a blue
background for strong advantages.

Although reference counting has normally had quite limited capabilities
and been quite tricky to apply for general linked data-structure
traversal, given a double-pointer-width compare-and-swap instruction,
it can work quite well, as shown in the ``Reference Counting with DCAS''
column.

As a rough rule of thumb, for best performance and scalability, you
should use RCU for read-intensive workloads and hazard pointers for
workloads that have significant update rates.
As another rough rule of thumb, a significant update rate has updates
as part of more than 10\% of its operations.
Reference counting with DCAS is well-suited for small systems and/or
low read-side contention, and particularly on systems that have limited
thread-local-storage capabilities.
Both RCU and reference counting with DCAS allow unconditional reference
acquisition.

Specialized workloads will have other considerations.
For example, small-memory multiprocessor systems might be best-served by
hazard pointers, while the read-mostly data structures in real-time
systems might be best-served by RCU.

The relationship between the Hazard Pointers proposal and this RCU
proposal is as follows:

\begin{enumerate}
\item	The \co{hazptr_obj_base} class is analogous to \co{rcu_obj_base}.
\item	The \co{hazptr_domain} class is analogous to \co{rcu_domain}.
\item	The private \co{hazptr_obj} class is analogous to the pre-existing
	\co{rcu_head} struct.
	Because this is a private hazard-pointers class, there is no
	need to have compatible names.
\item	There is no RCU class analogous to \co{hazptr_rec} because
	RCU does not track (or need to track) references to individual
	RCU-protected objects.
\item	There is no hazard pointers counterpart to the \co{rcu_guard}
	class.
	This is because hazard pointers does not have (or need)
	a counterpart to \co{rcu_read_lock()} and \co{rcu_read_unlock()}.
\end{enumerate}

\section{Summary}
\label{sec:Summary}

This paper demonstrates a way of creating C++ bindings for a C-language
RCU implementation, which has been tested against the userspace RCU
library.
Specifically, this document proposes
\co{rcu_domain} (Figure~\ref{fig:RCU Domain Base Class}),
\co{rcu_guard} (Figure~\ref{fig:RCU Guarded Readers}), and
\co{rcu_obj_base} (Figure~\ref{fig:RCU Callbacks: Derived-Type Approach}).
We believe that these bindings are also appropriate for the type-oblivious
C++ RCU implementations that information-hiding considerations are likely
to favor.

\section*{Acknowledgments}

We owe thanks to Pedro Ramalhete for his review and comments.
We are grateful to Jim Wasko for his support of this effort.

%\input{acknowledgments}
%\input{legal}

%\bibliographystyle{abbrv}
%\softraggedright

\bibliographystyle{acm}
\bibliography{bib/RCU,bib/WFS,bib/hw,bib/os,bib/parallelsys,bib/patterns,bib/perfmeas,bib/refs,bib/syncrefs,bib/search,bib/swtools,bib/realtime,bib/TM,bib/standards,bib/maze}

\section*{Change Log}
\label{sec:Change Log}

This paper first appeared as {\bf P0461R0} in October of 2016.
Revisions to this document are as follows:

\begin{itemize}
\item	Convert to single-column mode.
	(November 16, 2016.)
\item	Change \co{call()} to \co{retire()} for hazard-pointer compatibility.
	(January 4, 2017.)
\item	Change \co{rcu_scoped_reader} to \co{rcu_guard} for compatibility
	with existing RAII mechanisms.
	(January 19, 2017.)
\item	Change \co{rcu_head_delete} to \co{rcu_obj_base} for compatibility
	with hazard pointers.
	(January 19, 2017.)
\item	Update to indicate preferred C++ RCU approach.
	(January 19, 2017.)
\item	Call out relationships between classes for RCU and for hazard
	pointers.
	(January 19, 2017.)
\item	Add constructors to \co{rcu_domain} to match those of
	\co{hazptr_domain}.
	(February 1, 2017.)
\item	Add \co{quiescent_state_needed()} member function to
	\co{rcu_domain} to allow code using RCU to complain if
	its requirements are not met, based on discussions with
	Geoffrey Romer and Andrew Hunter.
	(February 3, 2017.)
\item	Added references to related papers.
	(February 5, 2017.)
\end{itemize}

At this point, the paper was published as {\bf P0461R1}.

\end{document}

%%%%%%%%%%%%%%%%%%%%%%%%%%%%%%%%%%%%%%%%%%%%%%%%%%%%%%%%%%%%%%%%%%%%%%%%%%%%%%
% for Ispell:
% LocalWords:  workingdraft BCM ednote SubSections xfig SubSection

