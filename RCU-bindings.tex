\documentclass[letterpaper,twocolumn,10pt]{article}
\usepackage{epsfig,endnotes}
%\usepackage{usenix,epsfig}

%%%%%%%%%%%%%%%%%%%%%%%%%%%%%%%%%%%%%%%%%%%%%%%%%%%%%%%%%%%%%%%%%%%%%%
\usepackage{subfig}
\usepackage{fixltx2e}
\usepackage{url}        % \url{} command with good linebreaks
\usepackage{amssymb,amsmath}
\usepackage{graphicx}
\usepackage{enumerate}
\usepackage{listings}
\usepackage{xspace}
\usepackage[table]{xcolor}
\lstset{basicstyle=\ttfamily}
\usepackage[bookmarks=true,bookmarksnumbered=true,pdfborder={0 0 0}]{hyperref}

% Avoid widows and orphans
\widowpenalty=500
\clubpenalty=500

%%%%%%%%%%%%%%%%%%%%%%%%%%%%%%%%%%%%%%%%%%%%%%%%%%%%%%%%%%%%%%%%%%%%%%

\begin{document}

\lstset{
	literate={\_}{}{0\discretionary{\_}{}{\_}}%
}
\newcommand{\co}[1]{\lstinline[breaklines=yes,breakatwhitespace=yes]{#1}}

\title{DNNNNR0: Proposed RCU C++ API}

\author{
{\bf Doc. No.: } WG21/DNNNNR0 \\
{\bf Date: } 2016-08-24 \\
{\bf Reply to: } Paul E. McKenney, Maged Michael, and Michael Wong \\
{\bf Email: } paulmck@linux.vnet.ibm.com, maged.michael@gmail.com, and
fraggamuffin@gmail.com
} % end author

% Use the following at camera-ready time to suppress page numbers.
% Comment it out when you first submit the paper for review.
%\thispagestyle{empty}

\pagestyle{myheadings}
\markright{WG21/DNNNNR0}

\maketitle

%%%%%%%%%%%%%%%%%%%%%%%%%%%%%%%%%%%%%%%%%%%%%%%%%%%%%%%%%%%%%%%%%%%%%%

This document is based on WG21/P0279R1 combined with feedback at
the 2015 Kona and 2016 Jacksonville meetings, which most notably called
for a C++-style method of handling different RCU implementations or
domains within a single translation unit, and which also contains
useful background material and references.
Unlike WG21/P0279R1, which simply introduced RCU's C-language practice,
this document presents proposals for C++-style RCU APIs.
At present, it appears that these are not conflicting proposals, but
rather ways of handling different C++ use cases resulting from
inheritance, templates, and different levels of memory pressure.

Note that this proposal is related to the hazard-pointer proposal in
that both proposals defer destructive actions such as reclamation until
all readers have completed.

Note also that a redefinition of the infamous \co{memory_order_consume}
is the subject of a separate paper.
% A detailed change log appears starting on
% page~\pageref{sec:Change Log}.

\section{Introduction}
\label{sec:Introduction}

This document proposes C++ APIs for read-copy update (RCU).

@@@ roadmap @@@

\section{Base RCU API}
\label{sec:Base RCU API}

\begin{figure}[tbp]
{ \scriptsize
\begin{verbatim}
 1 void std::rcu_read_lock();
 2 void std::rcu_read_unlock();
 3 void std::synchronize_rcu();
 4 void std::call_rcu(struct std::rcu_head *rhp,
 5                    void cbf(class rcu_head *rhp));
 6 void std::rcu_barrier();
 7 void std::rcu_register_thread();
 8 void std::rcu_unregister_thread();
 9 void std::rcu_quiescent_state();
10 void std::rcu_thread_offline();
11 void std::rcu_thread_online();
\end{verbatim}
}
\caption{Base RCU API}
\label{fig:Base RCU API}
\end{figure}

Figure~\ref{fig:Base RCU API}
shows the base RCU API as provided by implementations such as
userspace RCU~\cite{MathieuDesnoyers2009URCU,PaulMcKenney2013LWNURCU}.
This API is provided for compatibility with existing practice as
well as to provide the highest performance for the

Lines~1 and~2 show \co{rcu_read_lock()} and \co{rcu_read_unlock()},
which mark the beginning and the end, respectively, of an \emph{RCU read-side
critical section}.
These primitives may be nested, and matching \co{rcu_read_lock()}
and \co{rcu_read_unlock()} calls need not be in the same scope.
(That said, it is good practice to place them in the same scope
in cases where the entire critical section fits comfortably into
one scope.)

Line~3 shows \co{synchronize_rcu()}, which waits for any pre-existing
RCU read-side critical sections to complete.
The period of time that \co{synchronize_rcu()} is required to wait is
called a \emph{grace period}.
Note that a given call to \co{synchronize_rcu()} is \emph{not} required to
wait for critical sections that start later.

Lines~4 and~5 show \co{call_rcu()}, which, after a subsequent grace period
elapses, causes the \co{cbf(rhp)} \emph{RCU callback function} to be invoked.
Thus, \co{call_rcu()} is the asynchronous counterpart to
\co{synchronize_rcu()}.
In most cases, \co{synchronize_rcu()} is easier to use, however, \co{call_rcu()}
has the benefit of moving the grace-period delay off of the updater's
critical path.
Use of \co{call_rcu()} is thus critically important for good performance of
update-heavy workloads, as has been repeatedly discovered by any number of
people new to RCU~\cite{PaulEMcKenney2015ReadMostly}.

Note that although \co{call_rcu()}'s callbacks are guaranteed not to be
invoked too early, there is no guarantee that their execution won't be
deferred for a considerable time.
This can be a problem if a given program requires that all outstanding
RCU callbacks be invoked before that program terminates.
The \co{rcu_barrier()} function shown on line~6 is intended for this
situation.
This function blocks until all callbacks corresponding to previous
\co{call_rcu()} invocations have been invoked and also until after
those invocations have returned.
Therefore, taking the following steps just before terminating a program
will guarantee that all callbacks have completed:

\begin{enumerate}
\item	Take whatever steps are required to ensure that there are no
	further invocations of \co{call_rcu()}.
\item	Invoke \co{rcu_barrier()}.
\end{enumerate}

Carrying out this procedure just prior to program termination can be very
helpful for avoiding false positives when using tools such as valgrind.

Many RCU implementations require that every thread announce itself to
RCU prior to entering the first RCU read-side critical section, and
to announce its departure after exiting the last RCU read-side
critical section.
These tasks are carried out via the \co{rcu_register_thread()} and
\co{rcu_unregister_thread()}, respectively.

The implementations of RCU that feature the most aggressive implementations of
\co{rcu_read_lock()} and \co{rcu_read_unlock()} require that each thread
periodically pass through a \emph{quiescent state}, which is announce to RCU
using \co{rcu_quiescent_state()}.
A thread in a quiescent state is guaranteed not to be in an RCU
read-side critical section.
Threads can also announce entry into and exit from \emph{extended
quiescent states}, for example, before and after blocking system
calls, using \co{rcu_thread_offline()} and \co{rcu_thread_online()}.

\subsection{RCU Domains}
\label{sec:RCU Domains}

The userspace RCU library features several RCU implementations, each
optimized for different use cases.

The quiescent-state based reclamation (QSBR) implementation is intended
for standalone applications where the developers have full control
over the entire application, and where extreme read-side performance
and scalability is required.
Applications use \co{#include "urcu-qsbr.hpp"} to select QSBR and
\co{-lurcu -lurcu-qsbr} to link to it.
These applications must use \co{rcu_register_thread()} and
\co{rcu_unregister_thread()} to announce the coming and going
of each thread that is to execute \co{rcu_read_lock()} and
\co{rcu_read_unlock()}.
They must also use \co{rcu_quiescent_state()}, \co{rcu_thread_offline()},
and \co{rcu_thread_online()} to announce quiescent states to RCU.

The memory-barrier implementation is intended for applications that
can announce threads (again using \co{rcu_register_thread()} and
\co{rcu_unregister_thread()}), but for which announcing quiescent states is
impractical.
Such applications use \co{#include "urcu-mb.hpp"} and
\co{@@@} to select the memory-barrier implementation.
Such applications will incur the overhead of a full memory barrier in
each call to \co{rcu_read_lock()} and \co{rcu_read_unlock()}.

The signal-based implementation represents a midpoint between the QSBR
and memory-barrier implementations.
Like the memory-barrier implementation, applications must announce
threads, but need not announce quiescent states.
On the one hand, readers are almost as fast as in the QSBR implementation,
but on the other applications must give up a signal to RCU, by default
\co{SIGUSR1}.
Such applications use \co{#include "urcu-signal.hpp"} and
\co{@@@} to select signal-based RCU.

So-called ``bullet-proof RCU'' avoids the need to announce either threads
or quiescent states, and is therefore the best choice for use by
libraries that might well be linked with RCU-oblivious applications.
The penalty is that \co{rcu_read_lock()} incurs both a memory barrier
and a test and \co{rcu_read_unlock()} incurs a memory barrier.
Such applications or libraries use \co{#include urcu-bp.hpp} and
\co{@@@}.

\subsection{Run-Time Domain Selection}
\label{sec:Run-Time Domain Selection}

\begin{figure}[tbp]
{ \scriptsize
\begin{verbatim}
 1   class rcu_domain {
 2   public:
 3     virtual void register_thread() = 0;
 4     virtual void unregister_thread() = 0;
 5     virtual void read_lock() noexcept = 0;
 6     virtual void read_unlock() noexcept = 0;
 7     virtual void synchronize() noexcept = 0;
 8     virtual void call(class rcu_head *rhp,
 9           void cbf(class rcu_head *rhp)) = 0;
10     virtual void barrier() noexcept = 0;
11     virtual void quiescent_state() noexcept = 0;
12     virtual void thread_offline() noexcept = 0;
13     virtual void thread_online() noexcept = 0;
14   };
\end{verbatim}
}
\caption{RCU Domain Base Class}
\label{fig:RCU Domain Base Class}
\end{figure}

Figure~\ref{fig:RCU Domain Base Class}
shows the abstract base class for runtime selection of RCU domains.
Each domain creates a concrete subclass that implements its RCU APIs:

\begin{itemize}
\item	Bullet-proof RCU: \co{class rcu_bp}
\item	Memory-barrier RCU: \co{class rcu_mb}
\item	QSBR RCU: \co{class rcu_qsbr}
\item	Signal-based RCU: \co{class rcu_signal}
\end{itemize}

Of course, additional implementations of RCU may be constructed by
deriving from \co{rcu_domain} and/or by implementing the API
shown in
Figure~\ref{fig:Base RCU API}.

\section{Scoped Readers}
\label{sec:Scoped Readers}

\begin{figure}[tbp]
{ \scriptsize
\begin{verbatim}
 1   class rcu_scoped_reader {
 2   public:
 3     rcu_scoped_reader()
 4     {
 5       this->rd = nullptr;
 6       rcu_read_lock();
 7     }
 8
 9     rcu_scoped_reader(class rcu_domain *rd)
10     {
11       this->rd = rd;
12       rd->read_lock();
13     }
14
15     rcu_scoped_reader(const rcu_scoped_reader &) = delete;
16
17     ~rcu_scoped_reader()
18     {
19       if (this->rd)
20         this->rd->read_unlock();
21       else
22         rcu_read_unlock();
23     }
24
25   private:
26     class rcu_domain *rd;
27   };
\end{verbatim}
}
\caption{RCU Scoped Readers}
\label{fig:RCU Scoped Readers}
\end{figure}

In some cases, it might be convenient to use a scoped style for RCU readers,
especially if the read-side critical section might be exited via exception.
The \co{rcu_scoped_reader} class shown in
Figure~\ref{fig:RCU Scoped Readers}
may be used for this purpose.
An argumentless constructor uses the API, or an \co{rcu_domain} class
may be passed to the constructor to use the specified RCU
implementation.

\section{RCU Callback Handling}
\label{sec:RCU Callback Handling}

The traditional C-language RCU callback uses address arithmetic
to map from the \co{rcu_head} structure to the enclosing struct,
for example, via the \co{container_of()} macro.
Of course, this approach also works for C++, but this section first
looks at some approaches that leverage C++ overloading and inheritance.
This will not be an either-or situation: Several of these approaches
are likely to be generally useful.

\subsection{Derived Deletion}
\label{sec:Derived Deletion}

\begin{figure}[tbp]
{ \scriptsize
\begin{verbatim}
 1   template<typename T>
 2   class rcu_head_delete: public rcu_head {
 3   public:
 4     static void trampoline(rcu_head *rhp)
 5     {
 6       T *obj;
 7       rcu_head_delete<T> *rhdp;
 8
 9       rhdp = static_cast<rcu_head_delete<T> *>(rhp);
10       obj = static_cast<T *>(rhdp);
11       delete obj;
12     }
13
14     void call()
15     {
16       call_rcu(static_cast<rcu_head *>(this), trampoline);
17     }
18
19     void call(class rcu_domain &rd)
20     {
21       rd.call(static_cast<rcu_head *>(this), trampoline);
22     }
23   };
\end{verbatim}
}
\caption{RCU Callbacks: Derived Deletion}
\label{fig:RCU Callbacks: Derived Deletion}
\end{figure}

By far the most common RCU callback simply frees the data structure.
Figure~\ref{fig:RCU Callbacks: Derived Deletion}
shows the \co{rcu_head_delete} class, which supports this idiom
in cases where the RCU-protected data structure may inherit
from this class.

The \co{rcu_head_delete} class supplies a pair of overloaded \co{call()}
member functions, the of which takes no argument.
This argument-free member function arranges to \co{delete} the
object after a grace period, using \co{call_rcu()} to do so.

The second \co{call()} member function takes an \co{rcu_domain}
argument, and uses that domain's \co{call_rcu()} function to wait
for a grace period.

Use of this approach is quite straightforward.
For example, a class \co{foo} would inherit from
\co{rcu_head_delete<foo>}, and given a \co{foo} pointer \co{fp},
would execute \co{fp->call()} to cause the object referenced
by \co{fp} to be passed to \co{delete} at the end of a subsequent
grace period.
No further action is required.

However, it is sometimes necessary to do more than simply free an
object.
In many cases, additional teardown actions are required, and
it is often necessary to use a non-standard deallocator instead
of the C++ \co{delete}.

\subsection{Derived Function Call}
\label{sec:Derived Function Call}

\begin{figure}[tbp]
{ \scriptsize
\begin{verbatim}
 1   template<typename T>
 2   class rcu_head_derived: public rcu_head {
 3   public:
 4     static void trampoline(rcu_head *rhp)
 5     {
 6       T *obj;
 7       rcu_head_derived<T> *rhdp;
 8
 9       rhdp = static_cast<rcu_head_derived<T> *>(rhp);
10       obj = static_cast<T *>(rhdp);
11       if (rhdp->callback_func)
12         rhdp->callback_func(obj);
13       else
14         delete obj;
15     }
16
17     void call(void callback_func(T *obj) = nullptr)
18     {
19       this->callback_func = callback_func;
20       call_rcu(static_cast<rcu_head *>(this), trampoline);
21     }
22
23     void call(class rcu_domain &rd,
24         void callback_func(T *obj) = nullptr)
25     {
26       this->callback_func = callback_func;
27       rd.call(static_cast<rcu_head *>(this), trampoline);
28     }
29
30   private:
31     void (*callback_func)(T *obj);
32   };
\end{verbatim}
}
\caption{RCU Callbacks: Derived Function Call}
\label{fig:RCU Callbacks: Derived Function Call}
\end{figure}

The \co{rcu_head_derived} class provides overloaded \co{call()} methods,
as shown in
Figure~\ref{fig:RCU Callbacks: Derived Function Call}.
These methods take a function pointer or not and take an
\co{rcu_domain} class or not.
If supplied, the function is invoked after a grace period,
otherwise the object is handed to \co{delete}.
A supplied function might carry out teardown actions before freeing
the object, or it might be the deallocation function from a
specialized memory allocator.
A lambda may of course be supplied, but lambda capture is not supported.

If an \co{rcu_domain} is supplied, its \co{call_rcu()}
member function is used, otherwise the \co{call_rcu()} free
function is used.

Of course, the action of \co{rcu_head_delete} can be emulated by
omitting the function pointer, but in that case \co{rcu_head_delete}
uses less storage, since it need not store a function pointer.

\subsection{Pointer To Enclosing Class}
\label{sec:Pointer To Enclosing Class}

\begin{figure}[tbp]
{ \scriptsize
\begin{verbatim}
 1   template<typename T>
 2   class rcu_head_ptr: public rcu_head {
 3   public:
 4     rcu_head_ptr()
 5     {
 6       this->container_ptr = nullptr;
 7     }
 8
 9     rcu_head_ptr(T *containing_class)
10     {
11       this->container_ptr = containing_class;
12     }
13
14     static void trampoline(rcu_head *rhp)
15     {
16       T *obj;
17       rcu_head_ptr<T> *rhdp;
18
19       rhdp = static_cast<rcu_head_ptr<T> *>(rhp);
20       obj = rhdp->container_ptr;
21       if (rhdp->callback_func)
22         rhdp->callback_func(obj);
23       else
24         delete obj;
25     }
26
27     void call(void callback_func(T *obj) = nullptr)
28     {
29       this->callback_func = callback_func;
30       call_rcu(static_cast<rcu_head *>(this), trampoline);
31     }
32
33     void call(class rcu_domain &rd,
34         void callback_func(T *obj) = nullptr)
35     {
36       this->callback_func = callback_func;
37       rd.call(static_cast<rcu_head *>(this), trampoline);
38     }
39
40   private:
41     void (*callback_func)(T *obj);
42     T *container_ptr;
43   };
\end{verbatim}
}
\caption{RCU Callbacks: Pointer}
\label{fig:RCU Callbacks: Pointer}
\end{figure}

If complex inheritance networks make inheriting from an
\co{rcu_head} derived type impractical, one alternative is
to maintain a pointer to the enclosing class as shown in
Figure~\ref{fig:RCU Callbacks: Pointer}.
This \co{rcu_head_ptr} class is included as a member of the RCU-protected
class.
The \co{rcu_head_ptr} class's pointer must be initialized, for example,
in the RCU-protected class's constructor.

If the RCU-protected class is \co{foo} and the name of the
\co{rcu_head_ptr} member function is \co{rh}, then
\co{foo1.rh.call(my_cb)} would cause the function \co{my_cb()} to be
invoked after the end of a subsequent grace period.
As with the previous classes, omitting the function pointer results
in the object being passed to \co{delete} and an \co{rcu_domain}
object may be specified.

\subsection{Address Arithmetic}
\label{sec:Address Arithmetic}

\begin{figure}[tbp]
{ \scriptsize
\begin{verbatim}
 1   template<typename T>
 2   class rcu_head_container_of {
 3   public:
 4     static void set_field(const struct rcu_head T::*rh_field)
 5     {
 6       T t;
 7       T *p = &t;
 8
 9       rh_offset = ((char *)&(p->*rh_field)) - (char *)p;
10     }
11
12     static T *enclosing_class(struct rcu_head *rhp)
13     {
14       return (T *)((char *)rhp - rh_offset);
15     }
16
17   private:
18     static size_t rh_offset;
19   };
20
21   template<typename T>
22   size_t rcu_head_container_of<T>::rh_offset;
\end{verbatim}
}
\caption{RCU Callbacks: Address Arithmetic}
\label{fig:RCU Callbacks: Address Arithmetic}
\end{figure}

Figure~\ref{fig:RCU Callbacks: Address Arithmetic}
shows an approach that can be used if memory is at a premium and
the inheritance techniques cannot be used.
The \co{set_field()} method sets the offset of the
\co{rcu_head_container_of} member within the enclosing RCU-protected
structure, and the \co{enclosing_class()} member function
applies that offset to translate a pointer to the
\co{rcu_head_container_of} member to the enclosing RCU-protected structure.

\begin{figure*}[tbp]
{ \scriptsize
\begin{verbatim}
 1 void my_cb(struct std::rcu_head *rhp)
 2 {
 3   struct foo *fp;
 4
 5   fp = std::rcu_head_container_of<struct foo>::enclosing_class(rhp);
 6   std::cout << "Callback fp->a: " << fp->a << "\n";
 7 }
\end{verbatim}
}
\caption{RCU Callbacks: Address Arithmetic in Callback}
\label{fig:RCU Callbacks: Address Arithmetic in Callback}
\end{figure*}

This address arithmetic must be carried out in the callback function,
as shown in
Figure~\ref{fig:RCU Callbacks: Address Arithmetic in Callback}.

\section{Hazard Pointers and RCU: Which to Use?}
\label{sec:Hazard Pointers and RCU: Which to Use?}

\begin{table*}
\small
\centering
\begin{tabular}{l|p{1.5in}|p{1.5in}|p{1.5in}}
	& Reference Counting & RCU & Hazard Pointers \\
	\hline
	\hline
	Unreclaimed objects
		& \cellcolor{green!30}{\bf Bounded}
			& Unbounded
				& \cellcolor{green!30}{\bf Bounded} \\
	\hline
	Traversal forward progress
		& \cellcolor{green!30}{\bf Lock-free}
			& \cellcolor{blue!20}
			  {\bf Bounded population oblivious wait-free}
				& \cellcolor{green!30}{\bf Lock-free} \\
	\hline
	Reclamation forward progress $^*$
		& \cellcolor{blue!20}{\bf Bounded wait-free}
			& Blocking
				& \cellcolor{blue!20}{\bf Bounded wait-free} \\
	\hline
	Contention among readers
		& Can be very high
			& \cellcolor{green!30}{\bf No contention}
				& \cellcolor{green!30}{\bf No contention} \\
	\hline
	Traversal speed
		& Atomic read-modify-write updates
			& \cellcolor{green!30}{\bf No or low overhead}
				& Store-load fence \\
	\hline
	Reference acquisition
		& Conditional
			& \cellcolor{green!30}{\bf Unconditional}
				& Conditional \\
	\hline
	Automatic reclamation
		& \cellcolor{green!30}{\bf Yes}
			& No
				& No \\
	\hline
	Purpose of domains
		& N/A
			& Isolate long-latency readers
				& Enable type safety \\
\end{tabular}
\caption{Comparison of Deferred-Reclamation Mechanisms}
\label{tab:Comparison of Deferred-Reclamation Mechanisms}

\flushleft
\noindent
*~~Does not include memory allocator, just the reclamation itself.
\end{table*}

Table~\ref{tab:Comparison of Deferred-Reclamation Mechanisms}
provides a rough summary of the relative advantages of reference
counting, RCU, and hazard pointers.
Advantages are marked in bold with green background, or with a blue
background for strong advantages.

As a rough rule of thumb, for best performance and scalability, you
should use RCU for read-intensive workloads and hazard pointers for
workloads that have significant update rates.
As another rough rule of thumb, a significant update rate has updates
as part of more than 10\% of its operations.

Specialized workloads will have other considerations.
For example, small-memory multiprocessor systems might be best-served by
hazard pointers, while the read-mostly data structures in real-time
systems might be best-served by RCU.

\section{Summary}
\label{sec:Summary}

This paper demonstrates a way of creating C++ bindings for a C-language
RCU implementation, which has been tested against the userspace RCU
library.
We believe that these bindings are also appropriate for the type-oblivious
C++ RCU implementations that information-hiding considerations are likely
to favor.

\section*{Acknowledgments}

We owe thanks to Pedro Ramalhete for his review of P0232R0, which
greatly improved
Table~\ref{tab:Comparison of Deferred-Reclamation Mechanisms}.
We are grateful to Jim Wasko for his support of this effort.

%\input{acknowledgments}
%\input{legal}

%\bibliographystyle{abbrv}
%\softraggedright

\bibliographystyle{acm}
\bibliography{bib/RCU,bib/WFS,bib/hw,bib/os,bib/parallelsys,bib/patterns,bib/perfmeas,bib/refs,bib/syncrefs,bib/search,bib/swtools,bib/realtime,bib/TM,bib/standards,bib/maze}

% \section*{Change Log}
% \label{sec:Change Log}

% This paper first appeared as {\bf N4026} in May of 2014.
% Revisions to this document are as follows:

% \begin{itemize}
% \item	Mark the paper as a revision of N4036.
% 	(July 17, 2014.)
% \end{itemize}

% At this point, the paper was published as {\bf P0098R1}.

\end{document}

%%%%%%%%%%%%%%%%%%%%%%%%%%%%%%%%%%%%%%%%%%%%%%%%%%%%%%%%%%%%%%%%%%%%%%%%%%%%%%
% for Ispell:
% LocalWords:  workingdraft BCM ednote SubSections xfig SubSection

