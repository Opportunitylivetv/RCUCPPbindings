\documentclass[letterpaper,10pt]{article}
\usepackage{epsfig,endnotes}
%\usepackage{usenix,epsfig}

%%%%%%%%%%%%%%%%%%%%%%%%%%%%%%%%%%%%%%%%%%%%%%%%%%%%%%%%%%%%%%%%%%%%%%
\usepackage{subfig}
\usepackage{fixltx2e}
\usepackage{url}        % \url{} command with good linebreaks
\usepackage{amssymb,amsmath}
\usepackage{graphicx}
\usepackage{enumerate}
\usepackage{listings}
\usepackage{xspace}
\usepackage[table]{xcolor}
\usepackage{rotating}
\lstset{basicstyle=\ttfamily}
\usepackage[bookmarks=true,bookmarksnumbered=true,pdfborder={0 0 0}]{hyperref}

% Avoid widows and orphans
\widowpenalty=500
\clubpenalty=500

%%%%%%%%%%%%%%%%%%%%%%%%%%%%%%%%%%%%%%%%%%%%%%%%%%%%%%%%%%%%%%%%%%%%%%

\begin{document}

\lstset{
	literate={\_}{}{0\discretionary{\_}{}{\_}}%
}
\newcommand{\co}[1]{\lstinline[breaklines=yes,breakatwhitespace=yes]{#1}}

\title{D0461R2: Proposed RCU C++ API}

\author{
{\bf Doc. No.: } WG21/D0461R2 \\
{\bf Date: } 2017-07-13 \\
{\bf Reply to: } Paul E. McKenney, Maged Michael, Michael Wong,\\
Isabella Muerte, Arthur O'Dwyer, David Hollman, \\
Andrew Hunter, and Geoffrey Romer \\
{\bf Email: } paulmck@linux.vnet.ibm.com, maged.michael@gmail.com, \\
fraggamuffin@gmail.com, isabella.muerte@mnmlstc.com, \\
arthur.j.odwyer@gmail.com, dshollm@sandia.gov, \\
ahh@google.com, and gromer@google.com
} % end author

% Use the following at camera-ready time to suppress page numbers.
% Comment it out when you first submit the paper for review.
%\thispagestyle{empty}

\pagestyle{myheadings}
\markright{WG21/D0461R2}

\maketitle

%%%%%%%%%%%%%%%%%%%%%%%%%%%%%%%%%%%%%%%%%%%%%%%%%%%%%%%%%%%%%%%%%%%%%%

This document is based on WG21/P0279R1 combined with feedback at
the 2015 Kona, 2016 Jacksonville, 2016 Issaquah, 2017 Kona, and
2017 Toronto meetings, which most notably called
for a C++-style method of handling different RCU implementations or
domains within a single translation unit, and which also contains
useful background material and references.
Later feedback and evaluation of existing RCU implementations permitted
significant simplification.
Unlike WG21/P0279R1, which simply introduced RCU's C-language practice,
this document presents proposals for C++-style RCU APIs.
At present, it appears that these are not conflicting proposals, but
rather ways of handling different C++ use cases resulting from
inheritance, templates, and different levels of memory pressure.
This document also incorporates content from
WG21/P0232R0\cite{PaulEMcKennneyToolKitP0232R0}.

Note that this proposal is related to the hazard-pointer proposal in
that both proposals defer destructive actions such as reclamation until
all readers have completed.
See P0233R3, which updates ``P0233R2: Hazard Pointers:
Safe Resource Reclamation for Optimistic Concurrency''
at \url{http://www.open-std.org/jtc1/sc22/wg21/docs/papers/2016/p0233r2.pdf}.

This proposal is also related to ``P0561R0 An RAII Interface for Deferred
Reclamation'' at
\url{http://www.open-std.org/jtc1/sc22/wg21/docs/papers/2017/p0561r0.html}.
This RAII proposal has replaced the C++ wrapper APIs that appeared in
earlier revisions of this document.

Note also that a redefinition of the infamous \co{memory_order_consume}
is the subject of two separate papers:

\begin{enumerate}
\item	P0190R3, which updates
	``P0190R2: Proposal for New \co{memory_order_consume} Definition'',
	\url{http://www.open-std.org/jtc1/sc22/wg21/docs/papers/2016/p0190r2.pdf}.
\item	P0462R1, which updates
	``P0462R0: Marking \co{memory_order_consume} Dependency Chains'',
	\url{http://www.open-std.org/jtc1/sc22/wg21/docs/papers/2016/p0462r0.pdf}.
	Note however that P0462R1 is expected to be obsoleted by
	an alternative proposal by JF Bastien that has seen production
	use.
\end{enumerate}

Draft wording for this proposal may be found in the new working paper
``P0566R2: Proposed Wording for Concurrent Data Structures: Hazard
Pointer and Read-Copy-Update (RCU)''.

A detailed change log appears starting on
page~\pageref{sec:Change Log}.

\section{Introduction}
\label{sec:Introduction}

This document proposes C++ APIs for read-copy update (RCU).
For more information on RCU, including RCU semantics, see
WG21/P0462R0 (``Marking \co{memory_order_consume} Dependency Chains''),
WG21/P0279R1 (``Read-Copy Update (RCU) for C++''),
WG21/P0190R2 (``Proposal for New \co{memory_order_consume} Definition''),
and
WG21/P0098R1 (``Towards Implementation and Use of \co{memory_order_consume}'').

Specifically, this document proposes
\co{rcu_reader} (Figure~\ref{fig:RAII RCU Readers}),
\co{rcu_obj_base} (Figure~\ref{fig:Retiring RCU-Protected Objects}), and
several free functions
(Figures~\ref{fig:Retiring RCU-Protected Objects}
and~\ref{fig:RCU Updaters}).

Section~\ref{sec:Existing C-Language RCU API} presents the base (C-style) RCU API,
Section~\ref{sec:RAII RCU Readers} presents a proposal for scoped RCU readers,
Section~\ref{sec:Retiring RCU-Protected Objects} presents proposals for handling of
RCU callbacks,
Section~\ref{sec:Hazard Pointers and RCU: Which to Use?} presents a
table comparing reference counting, hazard pointers, and RCU, and finally
Section~\ref{sec:Summary} presents a summary.
This is followed by an informational-only appendix that shows
some alternatives that were considered and rejected.

A fully functional implementation is available on github:
\url{https://github.com/paulmckrcu/RCUCPPbindings}.
See the Test/ajodwyer2 directory: Other directories contain other
alternatives that were considered and rejected, although the efforts
of their respective authors are deeply appreciated.
Their work was a critically important part of the learning process
leading to the solution presented in this document.

\section{Existing C-Language RCU API}
\label{sec:Existing C-Language RCU API}

\begin{figure}[tbp]
{ \scriptsize
\begin{verbatim}
 1 void rcu_read_lock();
 2 void rcu_read_unlock();
 3 void synchronize_rcu();
 4 void call_rcu(struct rcu_head *rhp,
 5               void cbf(struct rcu_head *rhp));
 6 void rcu_barrier();
 7 void rcu_register_thread();
 8 void rcu_unregister_thread();
 9 void rcu_quiescent_state();
10 void rcu_thread_offline();
11 void rcu_thread_online();
\end{verbatim}
}
\caption{Existing C-Language RCU API}
\label{fig:Existing C-Language RCU API}
\end{figure}

Figure~\ref{fig:Existing C-Language RCU API}
shows the existing C-language RCU API as provided by implementations such as
userspace RCU~\cite{MathieuDesnoyers2009URCU,PaulMcKenney2013LWNURCU}.
This API is provided for compatibility with existing C-language practice as
well as to provide the highest performance for fast-path code.
As we will see in
Section~\ref{sec:Existing C-Language RCU API and C++},
the C++ user will not need to be concerned with most
of these API members.

\subsection{Existing C-Language RCU API Detailed Description}
\label{sec:Existing C-Language RCU API Detailed Description}

Lines~1 and~2 show \co{rcu_read_lock()} and \co{rcu_read_unlock()},
which mark the beginning and the end, respectively, of an \emph{RCU read-side
critical section}.
These primitives may be nested, and matching \co{rcu_read_lock()}
and \co{rcu_read_unlock()} calls need not be in the same scope.
(That said, it is good practice to place them in the same scope
in cases where the entire critical section fits comfortably into
one scope.)

Line~3 shows \co{synchronize_rcu()}, which waits for any pre-existing
RCU read-side critical sections to complete.
The period of time that \co{synchronize_rcu()} is required to wait is
called a \emph{grace period}.
Note that a given call to \co{synchronize_rcu()} is \emph{not} required to
wait for critical sections that start later.

Lines~4 and~5 show \co{call_rcu()}, which, after a subsequent grace period
elapses, causes the \co{cbf(rhp)} \emph{RCU callback function} to be invoked.
Thus, \co{call_rcu()} is the asynchronous counterpart to
\co{synchronize_rcu()}.
In most cases, \co{synchronize_rcu()} is easier to use, however, \co{call_rcu()}
has the benefit of moving the grace-period delay off of the updater's
critical path.
Use of \co{call_rcu()} is thus critically important for good performance of
update-heavy workloads~\cite{PaulEMcKenney2015ReadMostly}.

Note that although \co{call_rcu()}'s callbacks are guaranteed not to be
invoked too early, there is no guarantee that their execution won't be
deferred for a considerable time.
This can be a problem if a given program requires that all outstanding
RCU callbacks be invoked before that program terminates.
The \co{rcu_barrier()} function shown on line~6 is intended for this
situation.
This function blocks until all callbacks corresponding to previous
\co{call_rcu()} invocations have been invoked and also until after
those invocations have returned.
Therefore, taking the following steps just before terminating a program
will guarantee that all callbacks have completed:

\begin{enumerate}
\item	Take whatever steps are required to ensure that there are no
	further invocations of \co{call_rcu()}.
\item	Invoke \co{rcu_barrier()}.
\end{enumerate}

Carrying out this procedure just prior to program termination can be very
helpful for avoiding false positives when using tools such as valgrind.

Many RCU implementations require that every thread announce itself to
RCU prior to entering the first RCU read-side critical section, and
to announce its departure after exiting the last RCU read-side
critical section.
These tasks are carried out via the \co{rcu_register_thread()} and
\co{rcu_unregister_thread()}, respectively.

The implementations of RCU that feature the most aggressive implementations of
\co{rcu_read_lock()} and \co{rcu_read_unlock()} require that each thread
periodically pass through a \emph{quiescent state}, which is announced to RCU
using \co{rcu_quiescent_state()}.
A thread in a quiescent state is guaranteed not to be in an RCU
read-side critical section.
Threads can also announce entry into and exit from \emph{extended
quiescent states}, for example, before and after blocking system
calls, using \co{rcu_thread_offline()} and \co{rcu_thread_online()}.

\subsection{Existing C-Language RCU API and C++}
\label{sec:Existing C-Language RCU API and C++}

Because of recent advances in both userspace RCU and the Linux kernel's
support for userspace RCU, C++ users will not need to be concerned
with most of the C-language API members shown in
Figure~\ref{fig:Existing C-Language RCU API}.

Both \co{rcu_read_lock()} and \co{rcu_read_unlock()} are encapsulated
in a C++ RAII class named \co{rcu_reader}.
The \co{synchronize_rcu()} function is exposed via the
\co{std::synchronize_rcu()} static member function, and
the \co{rcu_barrier()} function is exposed via the
\co{std::rcu_barrier()} static member function.
The \co{call_rcu()} function is exposed via the
\co{std::rcu_obj_base<T,D>::retire()} member function,
and a non-intrusive variant of \co{call_rcu()} is exposed via the
\co{std::rcu_retire()} templated free function.

We expect that \co{rcu_register_thread()} and \co{rcu_unregister_thread()}
will be buried into the thread-creation and thread-exit portions of the
standard library.

Use of the \co{sys_membarrier()} RCU implementation from the userspace
RCU library allows \co{rcu_quiescent_state()}, \co{rcu_thread_offline()},
and \co{rcu_thread_online()} to be dispensed with.

Of course, implementation experience and use may result in changes
to the C++ API as well as to the userspace RCU library.

\section{RAII RCU Readers}
\label{sec:RAII RCU Readers}

\begin{figure}[tbp]
{ \scriptsize
\begin{verbatim}
 1   class std::rcu_reader {
 2   public:
 3     rcu_reader() noexcept;
 4     rcu_reader(std::defer_lock_t) noexcept;
 5     rcu_reader(const rcu_reader &) = delete;
 6     rcu_reader(rcu_reader &&other) noexcept;
 7     rcu_reader& operator=(const rcu_reader&) = delete;
 8     rcu_reader& operator=(rcu_reader&& other) noexcept;
 9     ~rcu_reader() noexcept;
10     void swap(rcu_reader& other) noexcept;
11     void lock() noexcept;
12     void unlock() noexcept;
13   };
\end{verbatim}
}
\caption{RAII RCU Readers}
\label{fig:RAII RCU Readers}
\end{figure}

The \co{rcu_reader} class shown in
Figure~\ref{fig:RAII RCU Readers}
may be used for RAII RCU read-side critical sections.
An argumentless constructor enters an RCU read-side critical section,
a constructor with an argument of type \co{defer_lock_t} defers
entering a critical section until a later call to a
\co{std::rcu_reader::lock()} member function,
and a constructor taking another \co{rcu_reader} instance
transfers the RCU read-side critical section from that instance
to the newly constructed instance.
The assignment operator may be used to transfer an RCU read-side critical
section from another \co{rcu_reader} instance to an already-constructed
instance.

This class is intended to be used in a manner similar to \co{std::lock_guard},
so the destructor exits the RCU read-side critical section.

This implementation is movable but not copyable, which enables
an RCU read-side critical section's scope to be arbitrarily
extended across function boundaries via \co{std::forward} and
\co{std::move}.
The \co{std::rcu_reader::swap()} member function swaps the roles
of a pair of \co{rcu_reader} instances in the expected manner.
Finally, the \co{std::rcu_reader::lock()} member function enters an
RCU read-side critical section and the \co{std::rcu_reader::unlock()}
member function exits its critical section.
Invoking \co{std::rcu_reader::lock()} on an instance already corresponding
to an RCU read-side critical section and invoking
\co{std::rcu_reader::unlock()} on an instance not corresponding to
an RCU read-side critical section results in undefined behavior.

\section{Retiring RCU-Protected Objects}
\label{sec:Retiring RCU-Protected Objects}

The traditional C-language RCU callback uses address arithmetic
to map from the \co{rcu_head} structure to the enclosing struct,
for example, via the \co{container_of()} macro.
Of course, this approach also works for C++, but this section describes
a more palatable approach that is quite similar to that proposed for
hazard pointers.
Other historical approaches may be found in
Sections~\ref{sec:Pointer To Enclosing Class}
and~\ref{sec:Address Arithmetic}.

\begin{figure}[tbp]
{ \scriptsize
\begin{verbatim}
 1 template<typename T, typename D = std::default_delete<T>>
 2 class std::rcu_obj_base {
 3 public:
 4         void retire(D d = {});
 5 };
 6
 7 template<typename T, typename D = std::default_delete<T>>
 8 void std::rcu_retire(T *p, D d = {});
\end{verbatim}
}
\caption{Retiring RCU-Protected Objects}
\label{fig:Retiring RCU-Protected Objects}
\end{figure}

The \co{rcu_obj_base} class provides a \co{retire()} method that
takes a deleter,
as shown in
Figure~\ref{fig:Retiring RCU-Protected Objects}.
This class contains any storage required by that deleter, so that
the caller of
\co{std::rcu_obj_base<T,D>::retire()} never needs to allocate
memory on the retire/free path.
The deleter's \co{operator()} is invoked after a grace period.
The deleter type defaults to \co{std::default_delete<T>},
but one could also use a
custom functor class with an \co{operator()} that carries out teardown actions
before freeing the object, or a raw function pointer type such as
\co{void(*)(T*)}.
Given sufficient type erasure and reconsitution, the \co{call_rcu()}
C-language free function from the userspace RCU library can be used to
implement \co{retire()}.

We recommend avoiding deleter types such as \co{std::function<void(T*)>}
(and also any other type requiring memory allocation) because
allocating memory on the free path can result in out-of-memory deadlocks,
but we nevertheless recognize that C++ applications that assume ample
memory might use such deleters for convenience.
However, such users are better served by the \co{std::rcu_retire()}
templated free function shown on lines~7-8 of
Figure~\ref{fig:Retiring RCU-Protected Objects}.
Although this function can allocate on the retire/free path, high-quality
implementations will take steps to ensure that such allocation is very
rare, for example, by pre-allocating sufficient storage to avoid
such allocation in the common case.
Simple implementations can instead provide a trivial \co{std::rcu_retire()}
function that is a thin wrapper around \co{std::rcu_obj_base::retire()}.

User experience has shown that high-quality implementations
of \co{std::rcu_retire()} can achieve better cache locality than can 
high-quality implementations of \co{std::rcu_obj_base::retire()},
which results in better performance and
scalability.\footnote{
	Note that this user experience is not limited to
	Google~\cite{MathieuDesnoyers2012URCU}.}
Users wishing the best performance and scalability will therefore tend
to prefer \co{std::rcu_retire()} over \co{std::rcu_obj_base::retire()}.

Finally, \co{std::rcu_retire()} is non-intrusive.
This means that (for example) an object of type \co{std::string} can be
passed to \co{std::rcu_retire()}.
In contrast, \co{std::rcu_obj_base::retire()} can only be passed types
related to \co{std::rcu_obj_base}.

\section{RCU Updaters}
\label{sec:RCU Updaters}

\begin{figure}[tbp]
{ \scriptsize
\begin{verbatim}
 1 void synchronize_rcu() noexcept;
 2 void rcu_barrier() noexcept;
\end{verbatim}
}
\caption{RCU Updaters}
\label{fig:RCU Updaters}
\end{figure}

RCU updaters can use the free functions shown in
Figure~\ref{fig:RCU Updaters}.
This class contains a \co{synchronize_rcu()} free function,
which maps to the C-language \co{synchronize_rcu()} function, which
waits for all pre-existing RCU readers to complete.
It also contains the \co{rcu_barrier()} free function,
which maps to the C-language \co{rcu_barrier()} function,
which waits for all pending \co{retire()} and \co{rcu_retire()}
deleters to be invoked.

\section{Hazard Pointers and RCU: Which to Use?}
\label{sec:Hazard Pointers and RCU: Which to Use?}

\begin{sidewaystable*}
\small
\centering
\begin{tabular}{p{1.2in}|p{1.2in}|p{1.2in}|p{1.2in}|p{1.2in}}
	& Reference Counting
		& \raggedright Reference Counting with DCAS
			& RCU
				& Hazard Pointers \\
	\hline
	\hline
	Unreclaimed objects
		& \cellcolor{green!30}{\bf Bounded}
			& \cellcolor{green!30}{\bf Bounded}
				& Unbounded
					& \cellcolor{green!30}{\bf Bounded} \\
	\hline
	\raggedright
	Contention among readers
		& Can be very high
			& Can be very high
				& \cellcolor{green!30}{\bf No contention}
					& \cellcolor{green!30}
					  {\bf No contention} \\
	\hline
	\raggedright
	Traversal forward progress
		& Either blocking or lock-free with limited reclamation
			& \cellcolor{green!30} {\bf Lock free}
				& \cellcolor{blue!20}
				  {\bf Bounded population oblivious wait-free}
					& \cellcolor{green!30}{\bf Lock-free} \\
	\hline
	\raggedright
	Reclamation forward progress $^*$
		& Either blocking or lock-free with limited reclamation
			& \cellcolor{green!30} {\bf Lock free}
				& Blocking
					& \cellcolor{blue!20}
					  {\bf Bounded wait-free} \\
	\hline
	Traversal speed
		& Atomic read-modify-write updates
			& Atomic read-modify-write updates
				& \cellcolor{green!30}{\bf No or low overhead}
					& Store-load fence \\
	\hline
	Reference acquisition
		& \cellcolor{green!30}{\bf Unconditional}
			& \cellcolor{green!30}{\bf Unconditional}
				& \cellcolor{green!30}{\bf Unconditional}
					& Conditional \\
	\hline
	\raggedright
	Automatic reclamation
		& \cellcolor{green!30}{\bf Yes}
			& \cellcolor{green!30}{\bf Yes}
				& No
					& No \\
	\hline
	Purpose of domains
		& N/A
			& N/A
				& Isolate long-latency readers
					& Limit contention, reduce space
					  bounds, etc.  \\
\end{tabular}
\caption{Comparison of Deferred-Reclamation Mechanisms}
\label{tab:Comparison of Deferred-Reclamation Mechanisms}

\flushleft
\noindent
*~~Does not include memory allocator, just the reclamation itself.
\end{sidewaystable*}

Table~\ref{tab:Comparison of Deferred-Reclamation Mechanisms}
provides a rough summary of the relative advantages of reference
counting, RCU, and hazard pointers.
Advantages are marked in bold with green background, or with a blue
background for strong advantages.

Although reference counting has normally had quite limited capabilities
and been quite tricky to apply for general linked data-structure
traversal, given a double-pointer-width compare-and-swap instruction,
it can work quite well, as shown in the ``Reference Counting with DCAS''
column.

As a rough rule of thumb, for best performance and scalability, you
should use RCU for read-intensive workloads and hazard pointers for
workloads that have significant update rates.
As another rough rule of thumb, a significant update rate has updates
as part of more than 10\% of its operations.
Reference counting with DCAS is well-suited for small systems and/or
low read-side contention, and particularly on systems that have limited
thread-local-storage capabilities.
Both RCU and reference counting with DCAS allow unconditional reference
acquisition.

Specialized workloads will have other considerations.
For example, small-memory multiprocessor systems might be best-served by
hazard pointers, while the read-mostly data structures in real-time
systems might be best-served by RCU.

The relationship between the Hazard Pointers proposal and this RCU
proposal is as follows:

\begin{enumerate}
\item	The \co{hazptr_obj_base} class is analogous to \co{rcu_obj_base}.
\item	There is no RCU counterpart to \co{hazptr_domain}, in part because
	RCU does not explicitly track read-side references to specific
	objects.
\item	The private \co{hazptr_obj} class is analogous to the pre-existing
	\co{rcu_head} struct used in many RCU implementations.
	Because this class is an implementation detail, there is no
	need to have compatible names.
\item	There is no RCU class analogous to \co{hazptr_rec} because
	RCU does not track (or need to track) references to individual
	RCU-protected objects.
\item	There is no hazard pointers counterpart to the \co{rcu_reader}
	class.
	This is because hazard pointers does not have (or need)
	a counterpart to \co{rcu_read_lock()} and \co{rcu_read_unlock()}.
\item	There is no hazard pointers counterpart to the \co{rcu_retire()}
	templated free function because no hazard-pointers user has
	expressed a need for it.
\end{enumerate}

\section{Summary}
\label{sec:Summary}

This paper demonstrates a way of creating C++ bindings for a C-language
RCU implementation, which has been tested against the userspace RCU
library.
Specifically, this document proposes
\co{rcu_reader} (Figure~\ref{fig:RAII RCU Readers}),
\co{rcu_obj_base} (Figure~\ref{fig:Retiring RCU-Protected Objects}), and
a few free functions (Figures~\ref{fig:Retiring RCU-Protected Objects}
and~\ref{fig:RCU Updaters}).
We believe that these bindings are also appropriate for the type-oblivious
C++ RCU implementations that information-hiding considerations are likely
to favor.

\section*{Acknowledgments}

We owe thanks to Pedro Ramalhete for his review and comments.
We are grateful to Jim Wasko for his support of this effort.

%\input{acknowledgments}
%\input{legal}

\appendix
\clearpage

This appendix contains historical proposals and directions.
As such, it is strictly informational.

\section{RCU Domains}
\label{sec:RCU Domains}

The userspace RCU library features several RCU implementations, each
optimized for different use cases.

\subsection{Compile-Time Domain Selection}
\label{sec:Compile-Time Domain Selection}

The quiescent-state based reclamation (QSBR) implementation is intended
for standalone applications where the developers have full control
over the entire application, and where extreme read-side performance
and scalability is required.
Applications use \co{#include "urcu-qsbr.hpp"} to select QSBR and
\co{-lurcu -lurcu-qsbr} to link to it.
These applications must use \co{rcu_register_thread()} and
\co{rcu_unregister_thread()} to announce the coming and going
of each thread that is to execute \co{rcu_read_lock()} and
\co{rcu_read_unlock()}.
They must also use \co{rcu_quiescent_state()}, \co{rcu_thread_offline()},
and \co{rcu_thread_online()} to announce quiescent states to RCU.

The memory-barrier implementation is intended for applications that
can announce threads (again using \co{rcu_register_thread()} and
\co{rcu_unregister_thread()}), but for which announcing quiescent states is
impractical.
Such applications use \co{#include "urcu-mb.hpp"} and
\co{-lurcu-mb} to select the memory-barrier implementation.
Such applications will incur the overhead of a full memory barrier in
each call to \co{rcu_read_lock()} and \co{rcu_read_unlock()}.

The signal-based implementation represents a midpoint between the QSBR
and memory-barrier implementations.
Like the memory-barrier implementation, applications must announce
threads, but need not announce quiescent states.
On the one hand, readers are almost as fast as in the QSBR implementation,
but on the other applications must give up a signal to RCU, by default
\co{SIGUSR1}.
Such applications use \co{#include "urcu-signal.hpp"} and
\co{-lurcu-signal} to select signal-based RCU.

So-called ``bullet-proof RCU'' avoids the need to announce either threads
or quiescent states, and is therefore the best choice for use by
libraries that might well be linked with RCU-oblivious applications.
The penalty is that \co{rcu_read_lock()} incurs both a memory barrier
and a test and \co{rcu_read_unlock()} incurs a memory barrier.
Such applications or libraries use \co{#include urcu-bp.hpp} and
\co{-lurcu-bp}.

\subsection{Run-Time Domain Selection}
\label{sec:Run-Time Domain Selection}

\begin{figure}[tbp]
{ \scriptsize
\begin{verbatim}
 1 class rcu_domain {
 2 public:
 3   constexpr explicit rcu_domain() noexcept { };
 4   rcu_domain(const rcu_domain&) = delete;
 5   rcu_domain(rcu_domain&&) = delete;
 6   rcu_domain& operator=(const rcu_domain&) = delete;
 7   rcu_domain& operator=(rcu_domain&&) = delete;
 8   virtual void register_thread() = 0;
 9   virtual void unregister_thread() = 0;
10   static constexpr bool register_thread_needed() { return true; }
11   virtual void quiescent_state() noexcept = 0;
12   virtual void thread_offline() noexcept = 0;
13   virtual void thread_online() noexcept = 0;
14   static constexpr bool quiescent_state_needed() { return false; }
15   virtual void read_lock() noexcept = 0;
16   virtual void read_unlock() noexcept = 0;
17   virtual void synchronize() noexcept = 0;
18   virtual void retire(rcu_head *rhp, void (*cbf)(rcu_head *rhp)) = 0;
19   virtual void barrier() noexcept = 0;
20 };
\end{verbatim}
}
\caption{RCU Domain Base Class}
\label{fig:RCU Domain Base Class}
\end{figure}

Figure~\ref{fig:RCU Domain Base Class}
shows the abstract base class for runtime selection of RCU domains.
Each domain creates a concrete subclass that implements its RCU APIs:

\begin{itemize}
\item	Bullet-proof RCU: \co{class rcu_bp}
\item	Memory-barrier RCU: \co{class rcu_mb}
\item	QSBR RCU: \co{class rcu_qsbr}
\item	Signal-based RCU: \co{class rcu_signal}
\end{itemize}

Of course, additional implementations of RCU may be constructed by
deriving from \co{rcu_domain} and/or by implementing the API
shown in
Figure~\ref{fig:Existing C-Language RCU API}.

\section{Historical RCU-Protected Retirement Plans}
\label{fig:Historical RCU-Protected Retirement Plans}

This section lists alternative methods of retiring RCU-protected
objects.
These might prove helpful for some use cases, and can be implemented
in terms of the intrusive approach described in
Section~\ref{sec:Retiring RCU-Protected Objects}.

\subsection{Pointer To Enclosing Class (Informational Only)}
\label{sec:Pointer To Enclosing Class}

\begin{figure}[tbp]
{ \scriptsize
\begin{verbatim}
 1   template<typename T>
 2   class rcu_head_ptr: public rcu_head {
 3   public:
 4     rcu_head_ptr()
 5     {
 6       this->container_ptr = nullptr;
 7     }
 8
 9     rcu_head_ptr(T *containing_class)
10     {
11       this->container_ptr = containing_class;
12     }
13
14     static void trampoline(rcu_head *rhp)
15     {
16       T *obj;
17       rcu_head_ptr<T> *rhdp;
18
19       rhdp = static_cast<rcu_head_ptr<T> *>(rhp);
20       obj = rhdp->container_ptr;
21       if (rhdp->callback_func)
22         rhdp->callback_func(obj);
23       else
24         delete obj;
25     }
26
27     void retire(void callback_func(T *obj) = nullptr)
28     {
29       this->callback_func = callback_func;
30       call_rcu(static_cast<rcu_head *>(this), trampoline);
31     }
32
33     void retire(class rcu_domain &rd,
34         void callback_func(T *obj) = nullptr)
35     {
36       this->callback_func = callback_func;
37       rd.retire(static_cast<rcu_head *>(this), trampoline);
38     }
39
40   private:
41     void (*callback_func)(T *obj);
42     T *container_ptr;
43   };
\end{verbatim}
}
\caption{RCU Callbacks: Pointer (Informational Only)}
\label{fig:RCU Callbacks: Pointer}
\end{figure}

If complex inheritance networks make inheriting from an
\co{rcu_head} derived type impractical, one alternative is
to maintain a pointer to the enclosing class as shown in
Figure~\ref{fig:RCU Callbacks: Pointer}.
This \co{rcu_head_ptr} class is included as a member of the RCU-protected
class.
The \co{rcu_head_ptr} class's pointer must be initialized, for example,
in the RCU-protected class's constructor.

If the RCU-protected class is \co{foo} and the name of the
\co{rcu_head_ptr} member function is \co{rh}, then
\co{foo1.rh.retire(my_cb)} would cause the function \co{my_cb()} to be
invoked after the end of a subsequent grace period.
As with the previous classes, omitting the deleter results
in the object being passed to \co{delete} and an \co{rcu_domain}
object may be specified.

Please note that this section is informational only: This approach
is \emph{not} being proposed for standardization.

\subsection{Address Arithmetic (Informational Only)}
\label{sec:Address Arithmetic}

\begin{figure}[tbp]
{ \scriptsize
\begin{verbatim}
 1   template<typename T>
 2   class rcu_head_container_of {
 3   public:
 4     static void set_field(const struct rcu_head T::*rh_field)
 5     {
 6       T t;
 7       T *p = &t;
 8
 9       rh_offset = ((char *)&(p->*rh_field)) - (char *)p;
10     }
11
12     static T *enclosing_class(struct rcu_head *rhp)
13     {
14       return (T *)((char *)rhp - rh_offset);
15     }
16
17   private:
18     static inline size_t rh_offset;
19   };
20
21   template<typename T>
22   size_t rcu_head_container_of<T>::rh_offset;
\end{verbatim}
}
\caption{RCU Callbacks: Address Arithmetic (Informational Only)}
\label{fig:RCU Callbacks: Address Arithmetic}
\end{figure}

Figure~\ref{fig:RCU Callbacks: Address Arithmetic}
shows an approach that can be used if memory is at a premium and
the inheritance techniques cannot be used.
The \co{set_field()} method sets the offset of the
\co{rcu_head_container_of} member within the enclosing RCU-protected
structure, and the \co{enclosing_class()} member function
applies that offset to translate a pointer to the
\co{rcu_head_container_of} member to the enclosing RCU-protected structure.

\begin{figure*}[tbp]
{ \scriptsize
\begin{verbatim}
 1 void my_cb(struct std::rcu_head *rhp)
 2 {
 3   struct foo *fp;
 4
 5   fp = std::rcu_head_container_of<struct foo>::enclosing_class(rhp);
 6   std::cout << "Callback fp->a: " << fp->a << "\n";
 7 }
\end{verbatim}
}
\caption{RCU Callbacks: Address Arithmetic in Callback (Informational Only)}
\label{fig:RCU Callbacks: Address Arithmetic in Callback}
\end{figure*}

This address arithmetic must be carried out in the callback function,
as shown in
Figure~\ref{fig:RCU Callbacks: Address Arithmetic in Callback}.

Please note that this section is informational only: This approach
is \emph{not} being proposed for standardization.

%\bibliographystyle{abbrv}
%\softraggedright

\bibliographystyle{acm}
\bibliography{bib/RCU,bib/WFS,bib/hw,bib/os,bib/parallelsys,bib/patterns,bib/perfmeas,bib/refs,bib/syncrefs,bib/search,bib/swtools,bib/realtime,bib/TM,bib/standards,bib/maze}

\section*{Change Log}
\label{sec:Change Log}

This paper first appeared as {\bf P0461R0} in October of 2016.
Revisions to this document are as follows:

\begin{itemize}
\item	Convert to single-column mode.
	(November 16, 2016.)
\item	Change \co{call()} to \co{retire()} for hazard-pointer compatibility.
	(January 4, 2017.)
\item	Change \co{rcu_scoped_reader} to \co{rcu_guard} for compatibility
	with existing RAII mechanisms.
	(January 19, 2017.)
\item	Change \co{rcu_head_delete} to \co{rcu_obj_base} for compatibility
	with hazard pointers.
	(January 19, 2017.)
\item	Update to indicate preferred C++ RCU approach.
	(January 19, 2017.)
\item	Call out relationships between classes for RCU and for hazard
	pointers.
	(January 19, 2017.)
\item	Add constructors to \co{rcu_domain} to match those of
	\co{hazptr_domain}.
	(February 1, 2017.)
\item	Add \co{quiescent_state_needed()} member function to
	\co{rcu_domain} to allow code using RCU to complain if
	its requirements are not met, based on discussions with
	Geoffrey Romer and Andrew Hunter.
	(February 3, 2017.)
\item	Added references to related papers.
	(February 5, 2017.)
\end{itemize}

At this point, the paper was published as {\bf P0461R1}.

Further revisions to this document are as follows:

\begin{itemize}
\item	Indicate which sections are preferred vs. informational.
	(February 17, 2017.)
\item	Update document number and boilerplate.
	(March 2, 2017.)
\item	Drop RCU domains, focusing on the \co{sys_membarrier()}
	implementation from the userspace RCU library.
\item	Provide \co{rcu_reader} as RAII class, replacing the old
	\co{rcu_guard}.
\item	Trim down the \co{rcu_obj_base} class.
\item	Remove implementation details from the code shown in
	the figures.
\item	Add a pointer to the github repository containing a
	working implementation.
\item	Add free functions for grace-period wait, retire-invocation
	wait, and non-intrusive retire.
\end{itemize}

\end{document}

%%%%%%%%%%%%%%%%%%%%%%%%%%%%%%%%%%%%%%%%%%%%%%%%%%%%%%%%%%%%%%%%%%%%%%%%%%%%%%
% for Ispell:
% LocalWords:  workingdraft BCM ednote SubSections xfig SubSection

